\documentclass[11pt]{article}
%\usepackage{graphicx}    % needed for including graphics e.g. EPS, PS
\usepackage{setspace}
\usepackage{amsfonts}
\usepackage{amsmath}
\usepackage{amssymb}
\topmargin -1.5cm        % read Lamport p.163
\oddsidemargin -0.04cm   % read Lamport p.163
\evensidemargin -0.04cm  % same as oddsidemargin but for left-hand pages
\textwidth 16.59cm
\textheight 21.94cm
%\pagestyle{empty}       % Uncomment if don't want page numbers
\parskip 7.2pt           % sets spacing between paragraphs
\renewcommand{\baselinestretch}{1} % Uncomment for 1.5 spacing between lines
\parindent 0pt         % sets leading space for paragraphs

\begin{document}
\title{COMS 331: Theory of Computing, Spring 2023\\
Homework Assignment 1}
\author{Due at 10:00PM, Wednesday, February 1, on Gradescope.\\}
\date{}
\maketitle

Note: In this class, 0 is a natural number, i.e. $0\in\mathbb{N}$.\\\\

{\bf Problem 1.} Prove or disprove: If $A = \{0^n1^n\mid n \in \mathbb{N}\}$, then $A^* = A$.\\\\

{\bf Problem 2.} Prove or disprove: If $B = \{x \in \{0,1\}^*\mid\#(0,x) = \#(1,x)\}$, then $B^* = B$.\\

Note: The notation $\#(0,x)$ is used to denote the number of 0's in $x$. Likewise, $\#(1,x)$ is used to denote the number of 1's in $x$.\\\\

{\bf Problem 3.} Prove: For every positive integer $n$,
\begin{center}
$\displaystyle\sum\limits_{k=1}^n \frac{1}{k^2} \leq 2 - \frac{1}{n}$.
\end{center}

\vspace{0.75cm}

The demonstration that all of mathematics can be carried out within the framework of set theory includes the following ``definition" of the natural numbers. First, the number 0 is defined to be $\varnothing$, the empty set. Next, for each previously defined natural number $n$, the number $n+1$ is defined to be the set $n \cup \{n\}$.\\

{\bf Problem 4.} (a) Write out the numbers 1, 2, and 3, defined as above.

(b) Prove: For every $n \in \mathbb{N}$, $n = \{k \in \mathbb{N} \mid k < n\}$.\\\\

{\bf Problem 5.} Prove: If $A = \{0,1\}$ and $B \subseteq \{0,1\}^*$, then
\begin{center}
$A^* = B^* \Rightarrow A \subseteq B$.
\end{center}

\vspace{0.75cm}

{\bf Problem 6.} Exhibit languages $A,B \subseteq \{0,1\}^*$ such that $A^* = B^*$ and $\{0,1\} \subseteq A \subsetneq B$.\\\\

{\bf Problem 7.} Define an (infinite) binary sequence $s \in \{0,1\}^{\infty}$ to be \emph{prefix-repetitive} if there are infinitely many strings $w \in \{0,1\}^*$ such that $ww \sqsubseteq s$.\\

Prove: If the bits of a sequence $s \in \{0,1\}^{\infty}$ are chosen by independent tosses of a fair coin, then
\begin{center}
$Prob[s$ is prefix-repetitive$] = 0$.
\end{center}
Note: $x \sqsubseteq y$ means that $x$ is a prefix of $y$ where $x$ is a string and $y$ is a string or sequence.\\


\end{document}
